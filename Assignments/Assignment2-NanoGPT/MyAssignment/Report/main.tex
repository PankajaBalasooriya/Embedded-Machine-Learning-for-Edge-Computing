\documentclass[fontsize=10pt]{article}
\usepackage[margin=0.70in]{geometry}
\usepackage{lipsum,mwe,abstract}
\usepackage[T1]{fontenc} 
\usepackage[english]{babel} 

\usepackage{fancyhdr} % Custom headers and footers
%\pagestyle{fancyplain} % Makes all pages in the document conform to the custom headers and footers
%\fancyhead{} 
%\fancyfoot[C]{\thepage} % Page numbering for right footer
\usepackage{lipsum}
\setlength\parindent{0pt} 

\usepackage{amsmath,amsfonts,amsthm} % Math packages
\usepackage{wrapfig}
\usepackage{graphicx}
\usepackage{float}
\usepackage{subcaption}
\usepackage{comment}
\usepackage{enumitem}
\usepackage{cuted}
\usepackage{sectsty} % Allows customizing section commands
\allsectionsfont{\normalfont \normalsize \scshape} % Section names in small caps and normal fonts

\renewenvironment{abstract} % Change how the abstract look to remove margins
 {\small
  \begin{center}
  \bfseries \abstractname\vspace{-.5em}\vspace{0pt}
  \end{center}
  \list{}{%
    \setlength{\leftmargin}{0mm}
    \setlength{\rightmargin}{\leftmargin}%
  }
  \item\relax}
 {\endlist}
 
\makeatletter
\renewcommand{\maketitle}{\bgroup\setlength{\parindent}{0pt} % Change how the title looks like
\begin{flushleft}
  \textbf{\@title}
  \@author \\ 
  \@date
\end{flushleft}\egroup
}
\makeatother

%% ------------------------------------------------------------------- 

\title{
\Large Title of the document  \\
[10pt] 
}
\date{\today}
\author{Name}

\begin{document}

 \maketitle

% --------------- ABSTRACT
\begin{abstract}
    \lipsum[1]
\end{abstract}

\rule{\linewidth}{0.5pt}

% --------------- MAIN CONTENT

\section{Introduction}
 \lipsum[2]
\begin{equation}  % Example of optimization equation
      \begin{tabular}{c c c}
           minimize & $f(x)$ & $x \in \mathbb{R} $\\
           subject to & $g(x)\leq 0 $&  \\
           & $h(x) = 0$& \\
      \end{tabular}
  \end{equation}
\lipsum[1]

\section{Objectives}
\lipsum[1-2]



\section{Methodology}
\lipsum[1-2]

\section{Results}
\lipsum[1-2]

\section{Conclusion}
\lipsum [4-5]


\begin{thebibliography}{}
\bibitem{Ref1}{First reference}
\bibitem{Ref2}{Second reference}
\bibitem{Ref3}{Third reference}
\end{thebibliography}


 \end{document}
 